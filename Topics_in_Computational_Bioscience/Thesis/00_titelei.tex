\begin{titlingpage}
\begin{center}
		
		
		% Oberer Teil der Titelseite:   
		
		
		
		\textsc{\centering The University of Oxford | The Department of Statistics}\\[0.2cm]
		% Title
		
		\newcommand{\HRule}{\rule{\linewidth}{0.5mm}}
		
		
		\begin{flushright}
			\HRule \\[.05cm]
		{  \huge \sc Graph Neural Networks for Molecular\setstretch{1} Property Prediction\par }
		\vspace*{.6cm}
		{\Large \sc Candidate Number:}\\[0.6cm]
		{\large In Fulfillment of Assessment for \\ 'Topics in Computational Biology'}\\[0.2cm]
		{\today}\\[.05cm]
		\HRule \\[0.4cm]
		\end{flushright}
		%\\[-.3cm]
		
	\end{center}
\begin{abstract}
	\noindent
Der Abstract fasst die zentralen Inhalte der Arbeit zusammen. Eine Wertung oder Interpretation erfolgt nicht. Dies hilft, sich einen groben Überblick über Fragestellung, Vorgehen und Ergebnisse zu verschaffen. Bestandteil sollen die Teile a) Hintergrundinformationen, Fragestellung, Zielsetzung, Forschungskontext, b) Methoden, c) Ergebnisse und d) Schlussfolgerungen, Anwendungsmöglichkeiten sein. Der Text ist knapp, vollständig und präzise, zudem objektiv und ohne persönliche Wertung. Achten Sie auf eine einfache und verständliche Sprache. Alle genannten Inhalte müssen auch im Hauptteil aufgegriffen werden. Den Inhalt objektiv und ohne persönliche Wertung wiedergeben. Gehen Sie auf die wichtigsten Konzepte, Resultate oder Folgerungen ein. Verwenden Sie keine Zitate und verzichten Sie auf Abkürzungen. In der Regel sind ca. 200 Wörter ausreichend.
\end{abstract}
\tableofcontents
\end{titlingpage}



\setcounter{tocdepth}{2}

