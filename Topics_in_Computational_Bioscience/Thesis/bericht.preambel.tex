% Dokumentklasse und Spracheinstellung
\documentclass[11pt, oneside, paper=A4, DIV=15, BCOR=0mm, abstract=true]{scrartcl}
\usepackage[english]{babel}
\usepackage[utf8]{inputenc}

% Schriftart
%LModern
%\usepackage{lmodern}
%Libertine
\usepackage{rawfonts}
\usepackage{oldlfont}
%\addtokomafont{disposition}{\rmfamily}
%\usepackage{libertinust1math}

% Kopf- und Fusszeilen mit scrlayer-scrpage steuern
% Trennungslinie oben und unten
\usepackage[headsepline]{scrlayer-scrpage}

%Section and subsection headings
\usepackage{titlesec}
\titleformat{\section}{\centering \sc}{\thesection.}{0.5em}{}
\titleformat{\subsection}[runin]{\bfseries}{\normalfont\thesubsection.}{0.5em}{}
\titleformat{\subsubsection}[runin]{\it}{\thesubsubsection.}{0.5em}{}
\titlespacing{\section}{0pt}{2.0ex plus .5ex
     minus .2ex}{1.0ex plus .2ex}



%Counter formatting
\renewcommand{\thesection}{\arabic{section}}
\renewcommand{\thesubsection}{\thesection.\arabic{subsection}}
\renewcommand{\thesubsubsection}{\thesubsection.\arabic{subsubsection}}
\renewcommand{\theparagraph}{\thesubsubsection.\alph{paragraph}}

\renewenvironment{abstract}{
    \section*{Abstract}
    \small
    }


%custom titlepage
\usepackage{titling}
\usepackage{setspace}

%tabel of contents formatting
\usepackage{titletoc}
\titlecontents{section}[1.5em]{}{\contentslabel{1.5em}}{\hspace*{-1.5em}}{\titlerule*[8pt]{.}\contentspage}
\usepackage[T1]{fontenc}
\usepackage{caption}
\captionsetup{font=footnotesize}



% Mathe, Symbole, EInheitendarstellung, Chemie
\usepackage{amsmath}
\usepackage{amsxtra}
\usepackage{eurosym}
\usepackage{siunitx}  
\sisetup{locale=DE}
\usepackage[version=4]{mhchem}

% Typographie
\usepackage[auto]{microtype}
\clubpenalty = 10000
\widowpenalty = 10000
\displaywidowpenalty = 10000

% Einbindung von Bildern, Tabellen, pdf-Seiten, Quellcode
\usepackage{graphicx}
\usepackage{multirow, multicol, booktabs}
\usepackage{threeparttable}
\usepackage{longtable}
\usepackage{rotating}
\usepackage{ltablex}
\usepackage{subfig}
\captionsetup[subtable]{position=top}
\usepackage{pdfpages}
\usepackage{listings}

% Darstellung von URL
\usepackage{url}
\urlstyle{same}

% Fussnoten, auch für Tabellen
\usepackage{footnote}
\makesavenoteenv{tabular} 

% Pakete für Kontrolle und Review
\usepackage{todonotes}
\usepackage{blindtext}

% Darstellung der Literaturangaben
\usepackage[
backend=bibtex,
style=trad-plain,
citestyle=numeric-comp,
maxbibnames=2,
firstinits=true
]{biblatex}

\renewcommand*{\labelnamepunct}{\addcolon\addspace}

% Speicherort der Literaturangaben (*.bib Datei)
\bibliography{literature/references}

% Fussnoten
% Markierung in der Fußnote selbst weder hochgestellt noch kleiner gesetzt
% \deffootnote{1em}{1em}{\thefootnotemark\ }
% linksbündige Fußnotenmarkierungen
\deffootnote{1.5em}{1em}{%
  \makebox[1.5em][l]{\thefootnotemark}%
}

% Fussnoten nicht umbrechen
\interfootnotelinepenalty=10000

% Gestaltung der Bildunterschrift und Tabellenüberschirften sowie Titelseitenangaben
\addtokomafont{caption}{\small}
\setkomafont{captionlabel}{\sffamily\bfseries}
\setkomafont{subject}{\Large\bfseries}
\setkomafont{author}{\normalfont}
\setkomafont{date}{\normalfont}
\setkomafont{publishers}{\normalfont}





%\renewcaptionname{ngerman}{\figurename}{Abb.}
%\renewcaptionname{ngerman}{\tablename}{Tab.} 

% Tabellenumgebungen mit Schriftgröße 10 und 7
\newenvironment{tabular10}{%
  \fontsize{10}{12}\selectfont\tabular
}{%
  \endtabular
}

\newenvironment{tabular7}{%
  \fontsize{7}{12}\selectfont\tabular
}{%
  \endtabular
}

% Verweise und Refernezen, pdf-Eisntellungen
% Angaben ggf. aktualisieren!
\usepackage[
pdftitle={Abschlussbericht},
pdfsubject={},
pdfauthor={},
pdfkeywords={},  
% Links nicht einrahmen
hidelinks
]{hyperref}
\usepackage[english]{cleveref}

%Kopfzeile
\usepackage{scrlayer-scrpage}
\setlength{\footskip}{50pt}