% Dokumentklasse und Spracheinstellung

\usepackage[english]{babel}
\usepackage[utf8]{inputenc}

% Schriftart
%LModern
%\usepackage{lmodern}
%Libertine
\usepackage{rawfonts}
\usepackage{oldlfont}
%\addtokomafont{disposition}{\rmfamily}
%\usepackage{libertinust1math}

% Kopf- und Fusszeilen mit scrlayer-scrpage steuern
% Trennungslinie oben und unten
%Section and subsection headings
\usepackage{titlesec}
%\titleformat{\section}{\centering \sc}{\thesection.}{0.5em}{}
%\titleformat{\subsection}[runin]{\bfseries}{\normalfont\thesubsection.}{0.5em}{}
%\titleformat{\subsubsection}[runin]{\it}{\thesubsubsection.}{0.5em}{}
%\titlespacing{\section}{0pt}{2.0ex plus .5ex
%	minus .2ex}{1.0ex plus .2ex}

\newcommand*\chem[1]{\ensuremath{\mathrm{#1}}}
\usepackage{chemfig}
\usepackage{blkarray}
\usepackage[ruled,vlined]{algorithm2e}
%Counter formatting
\renewcommand{\thesection}{\arabic{section}}
\renewcommand{\thesubsection}{\thesection.\arabic{subsection}}
\renewcommand{\thesubsubsection}{\thesubsection.\arabic{subsubsection}}
\renewcommand{\theparagraph}{\thesubsubsection.\alph{paragraph}}

\renewenvironment{abstract}{
	\section*{Abstract}
	\small
}


%custom titlepage
\usepackage{titling}
\usepackage{setspace}

%tabel of contents formatting
\usepackage{titletoc}
\titlecontents{section}[1.5em]{}{\contentslabel{1.5em}}{\hspace*{-1.5em}}{\titlerule*[8pt]{.}\contentspage}
\usepackage[T1]{fontenc}


% Mathe, Symbole, EInheitendarstellung, Chemie
\usepackage{amsmath}
\usepackage{amsxtra}
\usepackage{eurosym}
% Typographie
\usepackage[auto]{microtype}
\clubpenalty = 10000
\widowpenalty = 10000
\displaywidowpenalty = 10000

% Einbindung von Bildern, Tabellen, pdf-Seiten, Quellcode
\usepackage{graphicx}
\usepackage{booktabs,
	makecell, % for second example
	tabularx}



% Darstellung von URL
\usepackage{url}
\urlstyle{same}

% Fussnoten, auch für Tabellen
\usepackage{footnote}
\usepackage{tabularx}
% Pakete für Kontrolle und Review
\usepackage{todonotes}
\usepackage{blindtext}



\graphicspath{{./figures/}}


% Fussnoten
% Markierung in der Fußnote selbst weder hochgestellt noch kleiner gesetzt
% \deffootnote{1em}{1em}{\thefootnotemark\ }
% linksbündige Fußnotenmarkierungen

% Fussnoten nicht umbrechen
\interfootnotelinepenalty=10000





%\renewcaptionname{ngerman}{\figurename}{Abb.}
%\renewcaptionname{ngerman}{\tablename}{Tab.} 

% Tabellenumgebungen mit Schriftgröße 10 und 7


% Verweise und Refernezen, pdf-Eisntellungen
% Angaben ggf. aktualisieren!
%\usepackage[
%pdftitle={Abschlussbericht},
%pdfsubject={},
%pdfauthor={},
%pdfkeywords={},  
% Links nicht einrahmen
%hidelinks
%]{hyperref}
%\usepackage[english]{cleveref}

%\setlength{\footskip}{50pt}